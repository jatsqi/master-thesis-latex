\chapter{Implementation}
\label{chap:impl}
% 0.5 Seiten

	Building upon the algorithmic descriptions of Chapter \ref{chap:tree}, we want to discuss how such an algorithm can be efficiently implemented and practice and how the component is integrated into \ac{GCG} as a detector.
	The Chapter is divided into multiple sections, each describing a different aspect of the implementation:
	
	\begin{enumerate}
		\item Section \ref{chap:impl:architecture} contains a high-level overview about the architecture of the detector and its relation with other components in \ac{GCG}.
		\item In Sections \ref{chap:impl:architecture} - \ref{chap:impl:hashing}, we will give a brief overview about a few custom data structures required to implement the refinement more efficiently and how the tree is represented in memory.
		\item Sections \ref{chap:impl:cutoff} and \ref{chap:impl:multi} conclude the Chapter with two practical considerations:
		\begin{itemize}
			\item Is it possible to reduce to the number of candidate partitions before scoring?
			\item For larger and more complicated models, can we ensure a reasonable runtime?
		\end{itemize}
	\end{enumerate}
	
	This Chapter does not include actual numbers concerning space and runtime, we only look at the underlying concepts which are widely used in other algorithms to achieve e.g. a better practical runtime.
	
	\clearpage

	\section{Architecture}
	\label{chap:impl:architecture}
	% 2 Seiten
	
		\begin{figure}[ht!]
			\centering
			\includesvg[scale=0.7, inkscapelatex=false]{Bilder/PlantUML/out/comp/comp}
			\caption{text}
			\label{fig:impl:arch:overview}
		\end{figure}
		
		As mentioned in the introduction to Chapter \ref{chap:tree}, the approach is implemented as a \textit{Detector} in \ac{GCG}, contrary to the fact that the algorithm outputs a one or multiple partitions of constraints or variables.
		Generating such partitions is more aligned with the concept of a \textit{classifiers} from Section \ref{chap:gcg:classifiers}, but this comes with an important drawback.
		Because we want to incorporate information about variables for partitioning constraints and vice vrca for constraints, we have to delay the execution of the algorithm to a point in time \textit{after} the classification has finished, i.e., all relevant data is available.
	
	\section{Metadata}
	\label{chap:impl:meta}
	% 1 Seite
	
	\section{Data Structures}
	\label{chap:impl:structures}
	
	% 2 Seiten
	
	\section{Hashing}
	\label{chap:impl:hashing}
	% 1 Seite
	
	\section{Cutoff Conditions}
	\label{chap:impl:cutoff}
	
		\subsection{Local Conditions}
		% 1 Seite
		
		\subsection{Global Conditions}
		% 1 Seite
		
	
	\section{Multi-Threading}
	\label{chap:impl:multi}
	% 1 Seite

