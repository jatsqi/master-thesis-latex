% ------------------------------------------------------------
% LaTeX Template für die DHBW zum Schnellstart!
% Original: https://github.wdf.sap.corp/vtgermany/LaTeX-Template-DHBW
% ------------------------------------------------------------
% ---- Präambel mit Angaben zum Dokument
\documentclass[
	fontsize=12pt,           % Leitlinien sprechen von Schriftgröße 12.
	paper=A4,
	twoside=false,
	listof=totoc,            % Tabellen- und Abbildungsverzeichnis ins Inhaltsverzeichnis
	bibliography=totoc,      % Literaturverzeichnis ins Inhaltsverzeichnis aufnehmen
	titlepage,               % Titlepage-Umgebung anstatt \maketitle
	headsepline,             % horizontale Linie unter Kolumnentitel
	abstract,              % Überschrift einschalten, Abstract muss in {abstract}-Umgebung stehen
]{scrreprt}                  % Verwendung von KOMA-Report
\usepackage[utf8]{inputenc}  % UTF8 Encoding einschalten
\usepackage[english]{babel}  % Neue deutsche Rechtschreibung
\usepackage[T1]{fontenc}     % Ausgabe von westeuropäischen Zeichen (auch Umlaute)
\usepackage{microtype}       % Trennung von Wörtern wird besser umgesetzt
\usepackage{lmodern}         % Nicht-gerasterte Schriftarten (bei MikTeX erforderlich)
\usepackage{graphicx}        % Einbinden von Grafiken erlauben
\usepackage{wrapfig}         % Grafiken fließend im Text
\usepackage{setspace}        % Zeilenabstand \singlespacing, \onehalfspaceing, \doublespacing
\usepackage[
	%showframe,                % Ränder anzeigen lassen
	left=2.7cm, right=2.5cm,
	top=2.5cm,  bottom=2.5cm,
	includeheadfoot
]{geometry}                      % Seitenlayout einstellen
\usepackage{scrlayer-scrpage}    % Gestaltung von Fuß- und Kopfzeilen
\usepackage{acronym}             % Abkürzungen, Abkürzungsverzeichnis
\usepackage{titletoc}            % Anpassungen am Inhaltsverzeichnis
\contentsmargin{0.75cm}          % Abstand im Inhaltsverzeichnis zw. Punkt und Seitenzahl
\usepackage[                     % Klickbare Links (enth. auch "nameref", "url" Package)
  hidelinks,                     % Blende die "URL Boxen" aus.
  breaklinks=true                % Breche zu lange URLs am Zeilenende um
]{hyperref}
\usepackage[hypcap=true]{caption}% Anker Anpassung für Referenzen
\urlstyle{same}                  % Aktuelle Schrift auch für URLs
% Anpassung von autoref für Gleichungen (ergänzt runde Klammern) und Algorithm.
% Anstatt "Listing" kann auch z.B. "Code-Ausschnitt" verwendet werden. Dies sollte
% jedoch synchron gehalten werden mit \lstlistingname (siehe weiter unten).
%\addto\extrasngerman{%
%	\def\equationautorefname~#1\null{Gleichung~(#1)\null}
%	\def\lstnumberautorefname{Zeile}
%	\def\lstlistingautorefname{Listing}
%	\def\algorithmautorefname{Algorithmus}
%	% Damit einheitlich "Abschnitt 1.2[.3]" verwendet wird und nicht "Unterabschnitt 1.2.3"
%	% \def\subsectionautorefname{Abschnitt}
%}

% ---- Abstand verkleinern von der Überschrift
\renewcommand*{\chapterheadstartvskip}{\vspace*{.5\baselineskip}}

% Hierdurch werden Schusterjungen und Hurenkinder vermieden, d.h. einzelne Wörter
% auf der nächsten Seite oder in einer einzigen Zeile.
% LaTeX kann diese dennoch erzeugen, falls das Layout ansonsten nicht umsetzbar ist.
% Diese Werte sind aber gute Startwerte.
\widowpenalty10000
\clubpenalty10000

% ---- Für das Quellenverzeichnis
\usepackage[
	backend = biber,                % Verweis auf biber
	language = auto,
	style = numeric,                % Nummerierung der Quellen mit Zahlen
	sorting = none,                 % none = Sortierung nach der Erscheinung im Dokument
	sortcites = true,               % Sortiert die Quellen innerhalb eines cite-Befehls
	block = space,                  % Extra Leerzeichen zwischen Blocks
	hyperref = true,                % Links sind klickbar auch in der Quelle
	%backref = true,                % Referenz, auf den Text an die zitierte Stelle
	bibencoding = auto,
	giveninits = true,              % Vornamen werden abgekürzt
	doi=false,                      % DOI nicht anzeigen
	isbn=false,                     % ISBN nicht anzeigen
    alldates=short                  % Datum immer als DD.MM.YYYY anzeigen
]{biblatex}
\addbibresource{Inhalt/literatur.bib}
\setcounter{biburlnumpenalty}{3000}     % Umbruchgrenze für Zahlen
\setcounter{biburlucpenalty}{6000}      % Umbruchgrenze für Großbuchstaben
\setcounter{biburllcpenalty}{9000}      % Umbruchgrenze für Kleinbuchstaben
\DeclareNameAlias{default}{family-given}  % Nachname vor dem Vornamen
\AtBeginBibliography{\renewcommand{\multinamedelim}{\addslash\space
}\renewcommand{\finalnamedelim}{\multinamedelim}}  % Schrägstrich zwischen den Autorennamen
%\DefineBibliographyStrings{german}{
%  urlseen = {Einsichtnahme:},                      % Ändern des Titels von "besucht am"
%}
\usepackage[babel]{csquotes}


% ---- Für Mathevorlage
\usepackage{amsmath}    % Erweiterung vom Mathe-Satz
\usepackage{amssymb}    % Lädt amsfonts und weitere Symbole
\usepackage{MnSymbol}   % Für Symbole, die in amssymb nicht enthalten sind.


% ---- Für Quellcodevorlage
\usepackage{scrhack}                    % Hack zur Verw. von listings in KOMA-Script
\usepackage{listings}                   % Darstellung von Quellcode
\usepackage{xcolor}                     % Einfache Verwendung von Farben
% -- Eigene Farben für den Quellcode
\definecolor{JavaLila}{rgb}{0.4,0.1,0.4}
\definecolor{JavaGruen}{rgb}{0.3,0.5,0.4}
\definecolor{JavaBlau}{rgb}{0.0,0.0,1.0}
\definecolor{ABAPKeywordsBlue}{HTML}{6000ff}
\definecolor{ABAPCommentGrey}{HTML}{808080}
\definecolor{ABAPStringGreen}{HTML}{4da619}
\definecolor{PyKeywordsBlue}{HTML}{0000AC}
\definecolor{PyCommentGrey}{HTML}{808080}
\definecolor{PyStringGreen}{HTML}{008080}
% -- Farben für ABAP CDS
\definecolor{CDSString}{HTML}{FF8C00}
\definecolor{CDSKeywords}{HTML}{6000ff}
\definecolor{CDSAnnotation}{HTML}{00BFFF}
\definecolor{CDSComment}{HTML}{808080}
\definecolor{CDSFunc}{HTML}{FF0000}

% -- Default Listing-Styles

\lstset{
	% Das Paket "listings" kann kein UTF-8. Deswegen werden hier 
	% die häufigsten Zeichen definiert (ä,ö,ü,...)
	literate=%
		{á}{{\'a}}1 {é}{{\'e}}1 {í}{{\'i}}1 {ó}{{\'o}}1 {ú}{{\'u}}1
		{Á}{{\'A}}1 {É}{{\'E}}1 {Í}{{\'I}}1 {Ó}{{\'O}}1 {Ú}{{\'U}}1
		{à}{{\`a}}1 {è}{{\`e}}1 {ì}{{\`i}}1 {ò}{{\`o}}1 {ù}{{\`u}}1
		{À}{{\`A}}1 {È}{{\'E}}1 {Ì}{{\`I}}1 {Ò}{{\`O}}1 {Ù}{{\`U}}1
		{ä}{{\"a}}1 {ë}{{\"e}}1 {ï}{{\"i}}1 {ö}{{\"o}}1 {ü}{{\"u}}1
		{Ä}{{\"A}}1 {Ë}{{\"E}}1 {Ï}{{\"I}}1 {Ö}{{\"O}}1 {Ü}{{\"U}}1
		{â}{{\^a}}1 {ê}{{\^e}}1 {î}{{\^i}}1 {ô}{{\^o}}1 {û}{{\^u}}1
		{Â}{{\^A}}1 {Ê}{{\^E}}1 {Î}{{\^I}}1 {Ô}{{\^O}}1 {Û}{{\^U}}1
		{œ}{{\oe}}1 {Œ}{{\OE}}1 {æ}{{\ae}}1 {Æ}{{\AE}}1 {ß}{{\ss}}1
		{ű}{{\H{u}}}1 {Ű}{{\H{U}}}1 {ő}{{\H{o}}}1 {Ő}{{\H{O}}}1
		{ç}{{\c c}}1 {Ç}{{\c C}}1 {ø}{{\o}}1 {å}{{\r a}}1 {Å}{{\r A}}1
		{€}{{\euro}}1 {£}{{\pounds}}1 {«}{{\guillemotleft}}1
		{»}{{\guillemotright}}1 {ñ}{{\~n}}1 {Ñ}{{\~N}}1 {¿}{{?`}}1,
	breaklines=true,        % Breche lange Zeilen um 
	breakatwhitespace=true, % Wenn möglich, bei Leerzeichen umbrechen
	% Symbol für Zeilenumbruch einfügen
	prebreak=\raisebox{0ex}[0ex][0ex]{\ensuremath{\rhookswarrow}},
	postbreak=\raisebox{0ex}[0ex][0ex]{\ensuremath{\rcurvearrowse\space}},
	tabsize=4,                                 % Setze die Breite eines Tabs
	basicstyle=\ttfamily\small,                % Grundsätzlicher Schriftstyle
	columns=fixed,                             % Besseres Schriftbild
	numbers=left,                              % Nummerierung der Zeilen
	%frame=single,                             % Umrandung des Codes
	showstringspaces=false,                    % Keine Leerzeichen hervorheben
	keywordstyle=\color{blue},
	ndkeywordstyle=\bfseries\color{darkgray},
	identifierstyle=\color{black},
	commentstyle=\itshape\color{JavaGruen},   % Kommentare in eigener Farbe
	stringstyle=\color{JavaBlau},             % Strings in eigener Farbe,
	captionpos=b,                             % Bild*unter*schrift
	xleftmargin=5.0ex
}

% ---- Eigener JAVA-Style für den Quellcode
\renewcommand{\ttdefault}{pcr}               % Schriftart, welche auch fett beinhaltet
\lstdefinestyle{EigenerJavaStyle}{
	language=Java,                             % Syntax Highlighting für Java
	%frame=single,                             % Umrandung des Codes
	keywordstyle=\bfseries\color{JavaLila},    % Keywords in eigener Farbe und fett
	commentstyle=\itshape\color{JavaGruen},    % Kommentare in eigener Farbe und italic
	stringstyle=\color{JavaBlau}               % Strings in eigener Farbe
}

% ---- Eigener ABAP-Style für den Quellcode
\renewcommand{\ttdefault}{pcr}
\lstdefinestyle{EigenerABAPStyle}{
	language=[R/3 6.10]ABAP,
	morestring=[b]\|,                          % Für Pipe-Strings
	morestring=[b]\`,                          % für Backtick-Strings
	keywordstyle=\bfseries\color{ABAPKeywordsBlue},
	commentstyle=\itshape\color{ABAPCommentGrey},
	stringstyle=\color{ABAPStringGreen},
	tabsize=2,
	morekeywords={
		types,
		@data,
		as,
		lower,
		start,
		selection,
		order,
		by,
		inner,
		join,
		key,
		end,
		cast
	}
}

% ---- Eigener Python-Style für den Quellcode
\renewcommand{\ttdefault}{pcr}
\lstdefinestyle{EigenerPythonStyle}{
	language=Python,
	columns=flexible,
	keywordstyle=\bfseries\color{PyKeywordsBlue},
	commentstyle=\itshape\color{PyCommentGrey},
	stringstyle=\color{PyStringGreen}
}

%----- ABAP-CDS-View language
\lstdefinelanguage{ABAPCDS}{
	sensitive=false,
	%Keywords
	morekeywords={define,
		view,
		as,
		select,
		from,
		inner,
		join,
		on,
		key,
		case,
		when,
		then,
		else,
		end,
		true,
		false,
		cast,
		where,
		and,
		distinct,
		group,
		by,
		having,
		min,
		sum,
		max,
		count,
		avg
	},
	%Methoden
	morekeywords=[2]{
		div,
		currency\_conversion,
		dats\_days\_between,
		concat\_with\_space,
		dats\_add_days,
		dats\_is\_valid,
		dats\_add\_months,
		unit\_conversion,
		division,
		mod,
		abs,
		floor,
		ceil,
		round,
		concat,
		replace,
		substring,
		left,
		right,
		length
	},
	morecomment=[s][\color{CDSAnnotation}]{@}{:},
	morecomment=[l][\itshape\color{CDSComment}]{//},
	morecomment=[s][\itshape\color{CDSComment}]{/*}{*/},
	morestring=[b][\color{CDSString}]',
	keywordstyle=\bfseries\color{CDSKeywords},
	keywordstyle=[2]\color{CDSFunc}
}

  % Weitere Details sind ausgelagert

\usepackage{algorithm}                  % Für Algorithmen-Umgebung (ähnlich wie lstlistings Umgebung)
\usepackage{algpseudocode}              % Für Pseudocode. Füge "[noend]" hinzu, wenn du kein "endif",
                                        % etc. haben willst.

\makeatletter                           % Sorgt dafür, dass man @ in Namen verwenden kann.
                                        % Ansonsten gibt es in der nächsten Zeile einen Compilefehler.
%\renewcommand{\ALG@name}{Algorithmus}   % Umbenennen von "Algorithm" im Header der Listings.
\makeatother                            % Zeichen wieder zurücksetzen
\renewcommand{\lstlistingname}{Listing} % Erlaubt das Umbenennen von "Listing" in anderen Titel.

% ---- Tabellen
\usepackage{booktabs}  % Für schönere Tabellen. Enthält neue Befehle wie \midrule
\usepackage{multirow}  % Mehrzeilige Tabellen
\usepackage{siunitx}   % Für SI Einheiten und das Ausrichten Nachkommastellen
\sisetup{locale=DE, range-phrase={~bis~}, output-decimal-marker={,}} % Damit ein Komma und kein Punkt verwendet wird.
\usepackage{xfrac} % Für siunitx Option "fraction-function=\sfrac"

% ---- Für Definitionsboxen in der Einleitung
\usepackage{amsthm}                     % Liefert die Grundlagen für Theoreme
\usepackage[framemethod=tikz]{mdframed} % Boxen für die Umrandung
% ---- Definition für Highlight Boxen

% ---- Grundsätzliche Definition zum Style
%\newtheoremstyle{defi}
%  {\topsep}         % Abstand oben
%  {\topsep}         % Abstand unten
%  {\normalfont}     % Schrift des Bodys
%  {0pt}             % Einschub der ersten Zeile
%  {\bfseries}       % Darstellung von der Schrift in der Überschrift
%  {:}               % Trennzeichen zwischen Überschrift und Body
%  {.5em}            % Abstand nach dem Trennzeichen zum Body Text
%  {\thmname{#3}}    % Name in eckigen Klammern
%\theoremstyle{defi}

% ------ Definition zum Strich vor eines Texts
\newmdtheoremenv[
  hidealllines = true,       % Rahmen komplett ausblenden
  leftline = true,           % Linie links einschalten
  innertopmargin = 0pt,      % Abstand oben
  innerbottommargin = 4pt,   % Abstand unten
  innerrightmargin = 0pt,    % Abstand rechts
  linewidth = 3pt,           % Linienbreite
  linecolor = gray!40,       % Linienfarbe
]{defStrich}{Definition}     % Name der des formats "defStrich"

% ------ Definition zum Eck-Kasten um einen Text
\newmdtheoremenv[
  hidealllines = true,
  innertopmargin = 6pt,
  linecolor = gray!40,
  singleextra={              % Eck-Markierungen für die Definition
    \draw[line width=3pt,gray!50,line cap=rect] (O|-P) -- +(1cm,0pt);
    \draw[line width=3pt,gray!50,line cap=rect] (O|-P) -- +(0pt,-1cm);
    \draw[line width=3pt,gray!50,line cap=rect] (O-|P) -- +(-1cm,0pt);
    \draw[line width=3pt,gray!50,line cap=rect] (O-|P) -- +(0pt,1cm);
  }
]{defEckKasten}{Definition}  % Name der des formats "defEckKasten"  % Weitere Details sind ausgelagert

% ---- Für Todo Notes
\usepackage{todonotes}
\setlength {\marginparwidth }{2cm}      % Abstand für Todo Notizen

% ---- Zum Einbinden von PDF-Dokumenten
\usepackage{pdfpages}


% ---- Elektronische Version oder Gedruckte Version?
% ---- Unterschied: Die elektronische Version enthält keinen Platzhalter für die Unterschrift
\usepackage{ifthen}
\newboolean{e-Abgabe}
\setboolean{e-Abgabe}{false}    % false=gedruckte Fassung

% ---- Persönlichen Daten:
\newcommand{\titel}{Template \LaTeX\ Wiki von BAzubis für BAzubis}
\newcommand{\titelheader}{Titel welcher im Header auftaucht}
\newcommand{\arbeit}{Projektarbeit 1 (T3\_2000)}
\newcommand{\studiengang}{Informatik}
\newcommand{\studienjahr}{2015}
\newcommand{\autor}{Vorname Nachname}
\newcommand{\autorReverse}{Nachname, Vorname}
\newcommand{\verfassungsort}{Karlsruhe}
\newcommand{\matrikelnr}{0000000}
\newcommand{\kurs}{TINF15B1}
\newcommand{\bearbeitungsmonat}{Januar 2025}
\newcommand{\abgabe}{01. Februar 2025}
\newcommand{\bearbeitungszeitraum}{01.10.2024 - 31.01.2025}
\newcommand{\firmaName}{SAP SE}
\newcommand{\firmaStrasse}{Dietmar-Hopp-Allee 16}
\newcommand{\firmaPlz}{69190 Walldorf, Deutschland}
\newcommand{\betreuerFirma}{B-Vorname B-Nachname}
\newcommand{\betreuerDhbw}{DH-Vorname DH-Nachname}

% ---- Metainformation für das PDF Dokument
\hypersetup{
	pdftitle    = {\titel},
	pdfsubject  = {\arbeit},
	pdfauthor   = {\autor},
	%pdfkeywords = {Keywords angeben},
	pdfcreator  = {LaTeX},
	%pdfproducer = {in der Regel pdfTeX}
}

% ---- Definition der Kopf- und Fußzeilen
\clearpairofpagestyles                          % Löschen von LaTeX Standard
\automark[section]{chapter}                     % Füllen von section und chapter
\renewcommand*{\chaptermarkformat}{}            % Entfernt die Kapitelnummer
\renewcommand*{\sectionmarkformat}{}            % Entfernt die Sectionnummer
% Angaben [für "plain"]{für "scrheadings"}
\ihead[]{\titelheader}                          % Kopfzeile links
\chead[]{}                                      % Kopfzeile mitte
\ohead[]{\rightmark}                            % Kopfzeile rechts
\ifoot[]{}                                      % Fußzeile links
\cfoot*{\sffamily\pagemark}                     % Fußzeile mitte
\ofoot[]{}                                      % Fußzeile rechts
\KOMAoptions{
   headsepline = 0.2pt,                         % Liniendicke Kopfzeile
   footsepline = false                          % Liniendicke Fußzeile
}


% ---- Hilfreiches
\newcommand{\zB}{z.\,B. }   % "z.B." mit kleinem Leeraum dazwischen (ohne wäre nicht korrekt)
\newcommand{\dash}{d.\,h. }

\newcommand{\code}[1]{\texttt{#1}} % Ist einfacher zu schreiben als ständig \texttt und erlaubt
                                   % Änderungen im Nachhinein, wenn man z.B. Inline-Code anders stylen möchte.

% ---- Silbentrennung (falls LaTeX defaults falsch / nicht gewünscht sind)
\hyphenation{HANA}         % anstatt HA-NA
\hyphenation{Graph-Script} % anstatt GraphS-cript

% ---- Beginn des Dokuments
\begin{document}
\setlength{\parindent}{0pt}              % Keine Paragraphen Einrückung.
                                         % Dafür haben wir den Abstand zwischen den Paragraphen.
\setcounter{secnumdepth}{2}              % Nummerierungstiefe fürs Inhaltsverzeichnis
\setcounter{tocdepth}{1}                 % Tiefe des Inhaltsverzeichnisses. Ggf. so anpassen,
                                         % dass das Verzeichnis auf eine Seite passt.
\sffamily                                % Serifenlose Schrift verwenden.

% ---- Vorspann
% ------ Titelseite
\singlespacing
\thispagestyle{empty}
\begin{titlepage}
\enlargethispage{4cm}

\begin{figure}           % Logo vom Ausbildungsbetrieb und der DHBW
	% \vspace*{-5mm} % Sollte dein Titel zu lang werden, kannst du mit diesem "Hack" 
	%                  den Inhalt der Seite nach oben schieben.
	\begin{minipage}{0.49\textwidth}
		\flushleft
		%\includegraphics[height=2.5cm]{Bilder/Logos/Logo_SAP.pdf} 
	\end{minipage}
	\hfill
	\begin{minipage}{0.49\textwidth}
		\flushright
		%\includegraphics[height=2.5cm]{Bilder/Logos/Logo_DHBW.pdf} 
	\end{minipage}
\end{figure} 
\vspace*{0.1cm}

\begin{center}
	\huge{\textbf{\titel}}\\[1.5cm]
	\Large{\textbf{\arbeit}}\\[0.5cm]
	\normalsize{im Rahmen der Prüfung zum\\[1ex] \textbf{Master of Science (B.Sc.)}}\\[0.5cm]
	\Large{des Studienganges \studiengang}\\[1ex]
	\normalsize{an der Dualen Hochschule Baden-Württemberg Karlsruhe}\\[1cm]
	\normalsize{von}\\[1ex] \Large{\textbf{\autor}} \\[1cm]
	% Hinweis: Manche Dozenten möchten einen Hinweis auf den Sperrvermerk auf der Titelseite.
	% \large{{\color{red}- Sperrvermerk -}}\\[1cm]
\end{center}

\begin{center}
	\vfill
	\begin{tabular}{ll}
		Abgabedatum:                     & \abgabe \\[0.2cm]
		Bearbeitungszeitraum:            & \bearbeitungszeitraum \\[0.2cm]
		Matrikelnummer, Kurs:            & \matrikelnr , \kurs \\[0.2cm]
		Ausbildungsfirma:                & \firmaName \\
		                                 & \firmaStrasse \\
		                                 & \firmaPlz \\[0.2cm]
		Betreuer der Ausbildungsfirma:   & \betreuerFirma \\[0.2cm]
		Gutachter der Dualen Hochschule: & \betreuerDhbw \\[2cm]
	\end{tabular} 
\end{center}
\end{titlepage}
  % Titelseite
\newcounter{savepage}
\pagenumbering{Roman}                    % Römische Seitenzahlen
\onehalfspacing

% ------ Erklärung, Sperrvermerk, Abstact
\chapter*{Eidesstattliche Erklärung}
Ich versichere hiermit, dass ich meine \arbeit{} mit dem Thema:
\begin{quote}
	\textit{\titel}
\end{quote} 
gemäß § 5 der \enquote{Studien- und Prüfungsordnung DHBW Technik} vom 29. September 2017 selbstständig verfasst und keine anderen als die angegebenen Quellen und Hilfsmittel benutzt habe. Die Arbeit wurde bisher keiner anderen Prüfungsbehörde vorgelegt und auch nicht veröffentlicht.

\vspace{0.25cm}

Ich versichere zudem, dass die eingereichte elektronische Fassung mit der gedruckten Fassung übereinstimmt.

\vspace{1cm}

\verfassungsort, den \today \\[0.5cm]
\ifthenelse{\boolean{e-Abgabe}}
	{\underline{Gez. \autor}}
	{\makebox[6cm]{\hrulefill}}\\ 
\autorReverse

\renewcommand{\abstractname}{Abstract} % Veränderter Name für das Abstract
\begin{abstract}
\begin{addmargin}[1.5cm]{1.5cm}        % Erhöhte Ränder, für Abstract Look
\thispagestyle{plain}                  % Seitenzahl auf der Abstract Seite

\begin{center}
\small\textit{- English -}             % Angabe der Sprache für das Abstract
\end{center}

\vspace{0.25cm}

This is the starting point of the Abstract. For the final bachelor thesis, there must be an abstract included in your document. So, start now writing it in German and English. The abstract is a short summary with around 200 to 250 words.

\vspace{0.25cm}

Try to include in this abstract the main question of your work, the methods you used or the main results of your work.


\end{addmargin}
\end{abstract}
\renewcommand{\abstractname}{Abstract} % Veränderter Name für das Abstract
\begin{abstract}
\begin{addmargin}[1.5cm]{1.5cm}        % Erhöhte Ränder, für Abstract Look
\thispagestyle{plain}                  % Seitenzahl auf der Abstract Seite

\begin{center}
\small\textit{- Deutsch -}             % Angabe der Sprache für das Abstract
\end{center}

\vspace{0.25cm}

Dies ist der Beginn des Abstracts. Für die finale Bachelorarbeit musst du ein Abstract in deinem Dokument mit einbauen. So, schreibe es am besten jetzt in Deutsch und Englisch. Das Abstract ist eine kurze Zusammenfassung mit ca. 200 bis 250 Wörtern.

\vspace{0.25cm}

Versuche in das Abstract folgende Punkte aufzunehmen: Fragestellung der Arbeit, methodische Vorgehensweise oder die Hauptergebnisse deiner Arbeit.


\end{addmargin}
\end{abstract}

% ------ Inhaltsverzeichnis
\singlespacing
\tableofcontents

% ------ Verzeichnisse
\renewcommand*{\chapterpagestyle}{plain}
\pagestyle{plain}
\chapter*{Formelverzeichnis}
\addcontentsline{toc}{chapter}{Formelverzeichnis} % Hinzufügen zum Inhaltsverzeichnis 

% Definition des neuen Befehls für das Einfügen der Abkürzung der Einheit
\newcommand{\acrounit}[1]{
  \acroextra{\makebox[18mm][l]{\si[per-mode=fraction,fraction-function=\sfrac]{#1}}}
}
\begin{acronym}[dmin] % längstes Kürzel wird verw. für den Abstand zw. Kürzel u. Text

	% Alphabetisch selbst sortieren - nicht verwendete Formeln rausnehmen!
	% Allgemein: \acro{KÜRZEL}[ABKÜRZUNG]{\acrounit{SI-EINHEIT}BESCHREIBUNG}

	\acro{A}[\ensuremath{A}]{\acrounit{mm^2}Fläche}	
	\acro{D}[\ensuremath{D}]{\acrounit{mm}Werkstückdurchmesser}	
	\acro{dmin}[\ensuremath{d\textsubscript{min}}]{\acrounit{mm}kleinster Schaftdurchmesser}	
	\acro{L1}[\ensuremath{L\textsubscript{1}}]{\acrounit{mm}Länge des Werkstückes Nr. 1}	
	\acro{Fwinkel}[]{\acrounit{Grad}Freiwinkel}	
	\acro{Kwinkel}[]{\acrounit{Grad}Keilwinkel}

\end{acronym}

\chapter*{Abkürzungsverzeichnis}
\addcontentsline{toc}{chapter}{Abkürzungsverzeichnis} % Hinzufügen zum Inhaltsverzeichnis 

\begin{acronym}[WYSISWG] % längstes Kürzel wird verw. für den Abstand zw. Kürzel u. Text

	% Alphabetisch selbst sortieren - nicht verwendete Kürzel rausnehmen!
	\acro{AIR}{Adobe Integrated Runtime}
	\acro{AJAX}{Asynchronous Javascript and XML}
	\acro{ANSI}{American National Standards Institute}
	\acro{API}{Application Programming Interface}
	\acro{AR}{Augmented Reality}
	\acro{BAPI}{Business Application Programming Interface}
	\acro{BIOS}{Basic Input Output System}
	\acro{CDMA}{Code Division Multiple Access}
	\acro{HTTPS}{Hypertext Transfer Protocol Secure}
	\acro{ISBN}{Internationale Standardbuchnummer}
	\acrodefplural{ISBN}[ISBNs]{Internationale Standardbuchnummern}
	\acro{OLAP}{Online Analytical Processing}
	\acro{ORDBMS}{Object-Relational DataBase Management System}
	\acro{SDK}{Software Development Kit}
	\acro{SEO}{Search Engine Optimization}
	\acro{SSH}{Secure Shell}
	\acro{UEFI}{Unified Extensible Firmware Interface}
	\acro{USB}{Universal Serial Bus}
	\acro{VLAN}{Virtual Local Area Network}
	\acro{WYSISWG}{What You See Is What You Get}
	\acro{XSL}{Extensible Stylesheet Language}

\end{acronym}
\listoffigures                          % Erzeugen des Abbildungsverzeichnisses 
\listoftables                           % Erzeugen des Tabellenverzeichnisses
\renewcommand{\lstlistlistingname}{Quellcodeverzeichnis}
\lstlistoflistings                      % Erzeugen des Listenverzeichnisses
\setcounter{savepage}{\value{page}}


% ---- Inhalt der Arbeit
\cleardoublepage
\pagenumbering{arabic}                  % Arabische Seitenzahlen für den Hauptteil
\setlength{\parskip}{0.5\baselineskip}  % Abstand zwischen Absätzen
\rmfamily
\renewcommand*{\chapterpagestyle}{scrheadings}
\pagestyle{scrheadings}
\onehalfspacing
\chapter{Einleitung}

\section*{Information bezüglich des Inhaltsverzeichnisses}
Nachtrag zum Inhaltsverzeichnis: Dieses sollte wenn möglich nur eine Seite lang sein. Unterpunkte können dabei auch ausgelassen werden. Dies kann ganz einfach durchgeführt werden indem die section mit einen Stern geschrieben wird, wie bei der Sektion hier.

Des Weiteren solltest du beachten, dass du keine Unterpunkte alleine aufmachen darfst. Laut den DHBW Leitlinien sollte, wenn es einen Punkt 1.1 gibt, auch ein Punkt 1.2 existieren. Weitere Details dazu in den Leitlinien - zur Info: diese Vorlage hält sich aktuell nicht an diese Regelung, sie ist schließlich nur ein Template.

\section{Grobe Struktur der Arbeit}
Alle Details zur Struktur findest du in den Leitlinien ab Seite 21. Nachfolgend ist nur die verkürze Version mit allen wichtigen Punkten angegeben.

\subsection{Einleitung - was gibt es zu beachten?}
Die Einleitung sollte folgende Punkte beinhalten:
\begin{itemize}
\item \textbf{Gegenstand und Ziele der Arbeit \& Aufgabenbeschreibung, Einführung in Thema, Stand der Technik \& Forschung, Motivation der Aufgabenstellung \& Vorausblick}
\item Ausgangspunkt der Arbeit umreißen
\item Hinführung zur Problemstellung + Interesse des Lesers wecken
\item Allgemeine Einleitung ins Thema (keine Unternehmens-, Produktbeschreibungen oder Organigramme!)
\item Fragestellung präsentieren, Motivation erläutern
\item Randbedingungen und Betrachtungsgrenzen aufzeigen
\item Stand der Technik und aktuelle Lösungsfindung beschreiben, Vor- und Nachteile bisheriger Lösung anhand von Literatur darlegen
\end{itemize}

\subsection{Hauptteil - zeige was du kannst!}
Folgende Punkte kannst du in deinen Hauptteil einarbeiten:
\begin{itemize}
\item \textbf{Anforderungsdefinition, Anforderungsanalyse, Lösungsgenerierung, Lösungsbewertung, Umsetzung) in sinnvollen Gliederungspunkten}
\item Gewähltes Verfahren oder bestimmter Lösungsweg muss begründet werden
\item Bei Versuchen (nicht alle müssen genannt werden) müssen Voraussetzungen, Vernachlässigungen sowie Anordnung, Leistungsfähigkeit und Messgenauigkeit der einzelnen Versuchsanordnung angeben werden
\item Ergebnisse der Arbeit ausführlich diskutieren, Vergleiche mit Anschauungen und Erfahrungen vergleichen
\item Ziel der Arbeit: Eindeutige Folgerungen und Richtlinien für die Praxis
\end{itemize}

\subsection{Schluss - das Ziel ist nahe...}
Nachfolgende Aspekte können in den Schluss eingearbeitet werden:
\begin{itemize}
\item \textbf{Zusammenfassung und Ausblick}
\item Aufgabenstellung, Vorgehensweise, sowie wesentliche Ergebnisse kurz/präzise darstellen
\item Zusammenfassung eigenständig verständlich
\item Länge ca. 1-1,5 Seiten (Problem, Ziele, Vorgehensweise, Ergebnisse und Ausblick)
\end{itemize}

\subsection{Sonstige Tipps}
Es ist hilfreich, sich Notizen in das Dokument zu schreiben.
\LaTeX Kommentare haben jedoch den Nachteil, dass sie in der PDF nicht erscheinen.
Einfacher Text wird auch leicht überlesen.
Deshalb wird in dieser Vorlage das Package \code{todonotes} eingebunden, welches Notizen in dem Dokument sichtbar macht.\todo{So wie hier}
Auf der rechten Seite siehst du ein Beispiel.


% ---- Literaturverzeichnis
\cleardoublepage
\renewcommand*{\chapterpagestyle}{plain}
\pagestyle{plain}
\pagenumbering{Roman}                   % Römische Seitenzahlen
\setcounter{page}{\numexpr\value{savepage}+1}
\printbibliography[title=Literaturverzeichnis]

% ---- Anhang
\appendix
%\clearpage
%\pagenumbering{Roman}  % römische Seitenzahlen für Anhang

\newpage
\end{document}
